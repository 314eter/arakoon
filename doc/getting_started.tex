\documentclass[notitlepage]{article}
\usepackage{verbatim}
\begin{document}
\section{Getting Started}
All you need to run an arakoon node is one executable, and one configuration file. 
This executable can be used to start a node, run tests, or as an ad hoc client.
\subsection{Configuration}
The example below shows a configuration for a three node setup. 
As you can see, it's in the well known inifile format.

\begin{verbatim}
nodes = arakoon_0, arakoon_1,arakoon_2

#optional:
#master = arakoon_0
#lease_expiry = 60
#compression=true

# DANGER: only set this if you know what you're doing,
# and understand the risk.
# (in the 2 node case,
#  you might want to be able to continue
#  when 1 node is down,
#  so you lower the quorum to 1 iso the default)
# the downside is that if you do this on both sides of
# a split network you will end up having 2 realities.
#quorum = 1

[arakoon_0]
ip = 127.0.0.1
client_port = 4000
messaging_port = 4010
home = /tmp/arakoon_0
log_dir = /tmp/arakoon_0
# available levels are: debug info notice warning error fatal
log_level = debug

[arakoon_1]
ip = 127.0.0.1
client_port = 4001
messaging_port = 4011
home = /tmp/arakoon_1
log_dir = /tmp/arakoon_1
log_level = debug


[arakoon_2]
ip = 127.0.0.1
client_port = 4002
messaging_port = 4012
home = /tmp/arakoon_2
log_dir = /tmp/arakoon_2
log_level = debug
\end{verbatim}

\subsection{Starting a node}

\begin{verbatim}
./arakoon -config cfg/arakoon.ini -daemonize --node arakoon_0 
\end{verbatim}
This starts the node with name arakoon\_0 and daemonizes the process.


\subsection{Restarting a node}
If for whatever reason, a node dies, it can just be restarted. 
In normal situations it will just catchup from the other nodes and resume its responsibilities.
\paragraph{corrupt tlog}
If a tlog is corrupt, it can just be deleted. The node can be restarted afterwards, and will download the tlog from other nodes, update the database and resume its normal tasks.
\paragraph{corrupt database}
If, for whatever reason, a database is corrupt, one can delete the .db and .db.wal files.
Afterwards, the node can be restarted. 
It will fill the database from the tlogs, and resume its normal tasks
\subsection{Log rotation}
Sending SIGUSR1 to an arakoon process will cause it to close and reopen its log file.
This can be used in combination with logrotate to do whatever you want with the logs.

\subsubsection{tlog rotation}
Arakoon manages its own tlog rotation, every 100000 updates. 
After rotation, arakoon compresses the rolled out tlog. 
Tlogs can be dumped using the Arakoon executable.
\begin{verbatim}
./arakoon --dump-tlog <path_to_tlog_file>
\end{verbatim}

\end{document}
