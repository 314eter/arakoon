\subsection{nodes in ocaml}
Implementing the nodes in ocaml using Ocsigen's LWT library gives ample control over the fine grained concurrency we need through monadic coroutines.
\subsection{Pure Python client}
Besides the obvious ocaml client, Arakoon provides a pure python client as well. We have other in house projects that need it.

\subsection{other clients}
We've more or less catered for our own needs (Python,ocaml and C).
Other clients will have to implement our client-node protocol. 
Meanwhile, someone wrote an experimental Erlang client in less than a day's work. We're starting to run into problems because we have to maintain 3 clients instead of one.

\subsection{local key/value store}
We've picked tokyo cabinet. 
Our client interface matches its api quite well. 
It might be an idea to make this pluggable, but we don't need this at the moment

\subsection{forced master and quorum}
We needed to solve the case where you have only 2 nodes.
The simplest solution is to allow the master to be chosen by configuration and the quorum to be fixed. 
This way, you can chose for the 2 node case where you want the master, 
and if you're willing to take the risk to keep on writing in a case of a slave node not responding, you just set the quorum to 1.

\section{Other strategical decisions}
\subsection{REST interface}
We decided not to offer a REST interface.
\subsection{dynamically adding nodes}
All the nodes know one another from their configuration files,
but adding a node is not trivial. 
For example, one wants to add a third node in a two node setup and starts a node with a config referring to the two existing nodes. 
But these older nodes are clueless 
and have another opinion on how many nodes need to concur. 
A cluster protects itself by not answering to nodes it does not know; 
but this means adding a node means you need to restart the existing nodes.

\paragraph{}
It might be better to make the \emph{known\_nodes} a paxos value 
over which consensus must be reached, but this has additional risks.
The population can only change with reasonable increments.

