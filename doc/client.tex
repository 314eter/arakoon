\section{basic client interface}
\subsection{some notation}
Before we can descibe the client's interface, 
we need to introduce some notational tools to make it easier to convey things in a concise manner.

\begin{table}[h]
\begin{tabular}{|l|l|}
\hline
  notation & how to read it \\
\hline
  :                & has type \\
unit               & aka void  \\
a list             & a list of items of type a \\
a array            & a fixed size sequence of items of type a \\
a option           & either (Some a) or None \\
string             & char array \\

$ a \rightarrow b$ & a function from a to b \\
a C.t              & something that eventually yields something of type a \\
\hline
\end{tabular}
\end{table}

For example $ a : string $ just means that $a$ is a $string$;
$fibonacci : int \rightarrow int$ just means that fibonnacci is a function that takes an integer, and produces an integer as result;
$cat : string \rightarrow string \rightarrow string$ reads as 'cat is a function that takes 2 strings and yields a string as result'

$write: channel \rightarrow string \rightarrow unit\ C.t$ reads as 'write takes a channel and a string and eventually yields unit.


A client has a dictionary interface with some adjustments for latency: the functions have a return value of type $a\ C.t$.
\paragraph{type key = string}
Keys are strings.
There are no strict size limitations, which means that they such be small enough to be handled in their entirety
\paragraph{type value = string}
Values are strings too.
\paragraph{type update}
An update is either $Set(key,value)$ or $Delete(key)$.
\paragraph{exists : key $\rightarrow$ bool C.t} See if a value exists for a specific key, without retrieving it.
\paragraph{get : key $\rightarrow$ value C.t}
You can look up a value if you have the key. It will eventually yield either a value, or raise an exception.
\paragraph{set : key $\rightarrow$ value $\rightarrow$ unit C.t}
You can update a value for a key, regardless of current value (if any).
\paragraph{delete : key $\rightarrow$ unit C.t}
You can remove a key/value pair. There was a suggestion to open this op to allow a regular expression.
\paragraph{range : key option $\rightarrow$ bool $\rightarrow$
  key option $\rightarrow$ bool $\rightarrow$
  int $\rightarrow$ key list C.t}
\emph{range bkey binc ekey einc max} will yield a list of keys where
$max$ is the maximum number of keys (if $max < 0$ then you want them all).
The keys fall in the range \emph{kbey..ekey}.
\emph{binc} and \emph{einc} specify if the borders are included (\emph{true}) or not.
\paragraph{range\_entries: key option $\rightarrow$ bool $\rightarrow$
  key option $\rightarrow$ bool $\rightarrow$
  int $\rightarrow$ bool $\rightarrow$ (key * value) list C.t
}
will yield a list of key value pairs.
The parameters have the same semantics as for the range method.

\paragraph{test\_and\_set : key $\rightarrow$ value option $\rightarrow$ value option $\rightarrow$ (value option) C.t} This is a careful update.
\emph{test\_and\_set k expected new} only modifies the value to \emph{new} if the old value is \emph{expected}.
The resulting value will be returned.
Using a \emph{value option} instead of a value allows you to only set a value only if there was none for that key.
Using \emph{None} as new allows you to do a careful delete as well.

\paragraph{hello : string $\rightarrow$ string C.t}
The client identifies itself to the server, and the server replies with its version string.
\paragraph{who\_master: unit $\rightarrow$ string option C.t}
Allows the client to know which node currently acts as master.
If there is no master, or it is not known to this node, the result is None.
%
%\paragraph{last\_entries: int $\rightarrow$ out\_channel $\rightarrow$ unit C.t}
%allows the client to stream the transaction log into a channel, starting from entry $i$

\paragraph{sequence: update list $\rightarrow$ unit C.t}
Provides an atomic multi-update. Either all updates are performed or none. 
While this is not a full transaction, it provides enough functionality to safeguard consistency.
