\section{Implementation choices}
\subsection{Inter-node communication}
The nodes are fully connected with each other over tcp sockets.
The low level (de-)serialization is handled by our llio library (C,Python and ocaml implementations available).
This library is actually pretty efficient.
For example, on inteloids, fetching a 64 bit integer from a buffer boils down to a reinterpretation of 8 bytes in the buffer (in C, it merely is a cast, while in Python it's a \emph{struct.unpack('q',i)} ).
Nodes all know one another from their configuration.
Just above the socket layer, there's an abstract messaging layer, where you can just send and receive messages.



\subsection{Client-Node communication}
The client node communication has different needs, and hence a rpc like approach will be used. Following table describes what we do with primitives.
Just note that a list should be written out head first, so that naieve de-serialization will return the original.
\begin{table}[ht]
\begin{tabular}{|l|l|l|}
\hline
type  &   marshalled form & size (bytes)          \\
\hline
bool        & false $\rightarrow 0 | $ true $ \rightarrow 1   $ & 1 \\
int32       & little endian          & 4             \\
int64       & little endian          & 8             \\
string      & [size:$int_{32}$][bytes] & 4 + n     \\
float       & IEE754 double       & 8             \\
x option    & 0x00 (None) or 0x01 [x] & $|x| + 1$ \\
x list      & [size:$int_{32}$][$x_{n-1}$]\ldots[$x_{0}$]       &\\
x array     & [size:$int_{32}$][$x_{0}$]\ldots[$x_{n-1}$]       &\\
\hline 
1:int32     & 0x01 0x00 0x00 0x00         & \\
70000:int32 & 0x70 0x11 0x01 0x00         & \\
3.14:float  & 0x1f85eb51b81e0940          & \\
\hline 
Set(key,value) & [1:$int_{32}$][key:string][value:string] & \\
Delete(key)  & [2:$int_{32}$][key:string]                  & \\
Assert(k,vo) & [8:$int_{32}$][key:string][vo: string option] & \\
sequence     & [size(data):$int_{32}$][$bytes(data)$]      & \\

\hline
\end{tabular}
\end{table}

\subsubsection{protocol definition}
The protocol is a very simple request/response based binary protocol.
The client is the active party, and sends a command
\paragraph{successful rpc call}
\begin{equation*}
\begin{aligned}
client: \ & [command:int_{32}][parameter_0][parameter_1]\ldots & (\&flush) \\
server: \ & [0x0:int_{32}][result_0][result_1]\ldots & (\&flush)
\end{aligned}
\end{equation*}
\paragraph{call that failed}

\begin{equation*}
\begin{aligned}
client: \ & [command : int_{32}][parameter_0][parameter_1]\ldots & (\&flush) \\
server: \ & [rc:int_{32}][size:int_{32}][bytes] & (\&flush) \\
\end{aligned}
\end{equation*}

Each command is masked with the magic sequence $0xb1ff0000$.
The node checks the magic, proceeds with reading the parameters, and processes the request.
Then the response code and response are written.
If a failure happens, the server writes out a return code different from zero, and a string with a message after which he closes the connection.

The command codes and return codes are listed below.

\begin{table}[ht]
\begin{tabular}{|l|l|}
\hline
command & code \\
\hline
ping            & 0x0000 0001 \\
who\_master     & 0x0000 0002 \\
exists          & 0x0000 0007 \\
get             & 0x0000 0008 \\
set             & 0x0000 0009 \\
delete          & 0x0000 000a \\
range           & 0x0000 000b \\
prefix\_keys    & 0x0000 000c \\
test\_and\_set  & 0x0000 000d \\
last\_entries   & 0x0000 000e \\
range\_enties   & 0x0000 000f \\
sequence        & 0x0000 0010 \\
multi\_get      & 0x0000 0011 \\
expect\_progress\_possible & 0x0000 0012 \\
user\_function  & 0x0000 0015 \\
assert          & 0x0000 0016 \\
get\_key\_count & 0x0000 001a \\
confirm         & 0x0000 001b \\
\hline
\end{tabular}
\end{table}

\begin{table}[ht]
\begin{tabular}{|l|l|}
\hline
return code & condition \\
\hline
0x0000 0000  & success!  \\
0x0000 0001  & command has no magic  \\
0x0000 0002  & too many dead nodes \\
0x0000 0003  & no hello        \\
0x0000 0004  & not master      \\
0x0000 0005  & not found       \\
0x0000 0006  & wrong cluster   \\
0x0000 0007  & assertion failed \\
\ldots       & \ldots          \\
0x0000 00ff  & unknown failure \\
\end{tabular}
\end{table}
